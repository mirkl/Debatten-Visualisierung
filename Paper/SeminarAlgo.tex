\documentclass[a4paper, oneside]{scrartcl}

% Zeichensatz und Sprachpaket
\usepackage[utf8]{inputenc}
\usepackage[ngerman]{babel} 
\usepackage{arev}
\usepackage[T1]{fontenc}

% Seite anpassen
\usepackage{geometry}	
\geometry{a4paper, top=12mm, left=18mm, right=18mm, bottom=12mm,
headsep=10mm, footskip=12mm}

% Einrücken eines neues Absatzes auf 0
\setlength{\parindent}{0pt}
\setlength{\parskip}{6pt}	
\usepackage{titlesec}
\titlespacing{\section}{0pt}{1em}{0.3em}

% Unterpunkte in Aufzählungen andere Symbole geben
%\usepackage{pifont}
%\renewcommand{\labelitemi}{\ding{104}}
%\renewcommand{\labelitemii}{\ding{105}}
\renewcommand*\labelitemii{$\circ$}
%\renewcommand{\thesection}{\Alph{section}} 

% Farben
\usepackage{color}
\usepackage[usenames,dvipsnames, table]{xcolor}

% Um Bilder einzufügen
\usepackage{graphicx}
%\usepackage{tabularx}

% Abstand zwischen Aufzählungen kleiner definieren 
%\let\origitemize\itemize	
%\def\itemize{\origitemize\itemsep0pt}

% Für TODOs
\usepackage{todonotes}

% Quellen

%\bibliographystyle{IEEEtranSA}	% english style
\bibliographystyle{babalpha-fl}	% german style 


% Formelzeugs
\usepackage{amsfonts}   
\usepackage{amssymb}
\usepackage{amsmath}
\def\R{\ensuremath{\mathbb{R}}}
\def\RR{\ensuremath{\mathcal{R}}}
\def\C{\ensuremath{\mathbb{C}}}
\def\CC{\ensuremath{\mathcal{C}}}
\def\Z{\ensuremath{\mathbb{Z}}}
\def\N{\ensuremath{\mathbb{N}}}
\def\F{\ensuremath{\mathbb{F}}}
\def\dann{\ensuremath{\Rightarrow}}
\def\gdw{\ensuremath{\Leftrightarrow}}
\def\tr{\ensuremath{\text{tr}}}
\def\deg{\ensuremath{\text{deg}}}
\def\Aut{\ensuremath{\text{Aut}}}
\def\Abb{\ensuremath{\text{Abb}}}

\usepackage{ntheorem}
\theorembodyfont{\normalfont}
\theoremstyle{changebreak} 
\newtheorem{theorem}{Theorem}
\newtheorem{lemma}{Lemma}
\newtheorem*{defi}{Definition}
\newtheorem*{problem}{Problem}









\author{Christian, Jonathan, Michael, Mira, Sebastian, Sven}
\date{\today}

% Inhaltabhängig! 
\title{\vspace{3cm}Inkrementeller, kräftebasierter Layoutalgorithmus für hierarchische Argumentkarten}

%--BEGIN--BEGIN--BEGIN--BEGIN--BEGIN--BEGIN--BEGIN--BEGIN-- 
\begin{document}   
%--BEGIN--BEGIN--BEGIN--BEGIN--BEGIN--BEGIN--BEGIN--BEGIN-- 
\nonfrenchspacing	
\maketitle

\begin{center}
\vspace{-9cm}
\includegraphics[width = 0.3\linewidth]{Pics/KITLogo.jpg}
\vspace{6.5cm}
\end{center}

\listoftodos

\todo[inline]{Alles}
\section{Einleitung}
% Darstellung von Argumentkarten
% - Was sind Argumentkarten?
% - Wieso will man sie darstellen?

\section{Problemstellung}
% Layoutproblem
% - Gegeben/Eingabe: 
	% Graph mit Knoten V, welche Größe besitzen, und gerichteten gefärbten Kanten, 
	% sowie azyklische hierarchische Struktur von Gruppen von Knoten und Gruppen
	% Anfangslayout (!?)
% - Vorgaben an Layout: 
	% Zeichenkonventionen, hard constraints:
	% - Knoten und Knoten überschneiden sich nicht (jedoch dürfen Kanten Gruppen schneiden, wenn sie zu Element der Gruppe gehören)
	% - Gruppen umschließen alle ihre Elemente (welche zusammengeklappt halt in einem Punkt zusammenfallen)
	% Ästhetikkritieren
	% - Kreuzungsminimierung
	% - gleichmäßige Kantenlängen (?)
	% - Wendepunktminimierung von Kanten
	% - Struktur hervorheben in Gruppen (?)
	% Lokale Nebenbedingungen
	% - Nummerierung von Nachbarknoten im Layout repräsentieren
	

% Verhalten beim Öffnen oder Schließen von Gruppen
% - Vorgaben an Verhalten: 
	% Ästhetikkritieren
	% - relative Positionen von Knoten und Gruppen verändern sich nur wenig
	% - Stuktur vor Zustandsänderung bleibt ähnlich und wiedererkennbar
	
% Gewählter Zeichenstil
% - Argumente als Rechtecke (mit fester Breite?)
% - Gruppen als Kreise
% - Kanten als ``glatte'' Kurven (bis zu welchem Grad?)
% - Kanten in Gruppen rein/aus Gruppe raus zu/von inneren Elementen nur über Ports


\section{Algorithmus}
% Idee für Layout: 
% - Gruppenelemente in Blase passen
% - festes Layout pro Hierarchiestufe berechnen
% - Layouts in Gruppen richten sich nach Ports aus Layout von Stufe darüber
% - Layouts niedrigerer Stufe beeinflussen Layouts höherer Stufe nur durch benötigten Platz
	
		% bottom-up und top-down
	% (Layoutalgorithmus hierbei egtl frei wählbar, kräftebasiert aber besser in Hinsicht auf Verhalten Algorithmus mit Ankern)
	% bottom-up
		% Berechne Layouts pro Gruppe auf niederigster Stufe
		% versuche hierbei Fläche klein zu halten bzw Elemente in Kreis zu zwingen
		% Berechne die approximierte resultierende benötigte Fläche für Gruppe
	
		% Benutze appr. benötigte Fläche für Wiederholung des Schritts auf Stufe höher (bis auf letzte)

	% top-down
		% (hier noch zu entscheiden ob so wie beschrieben oder mit Gruppen offen)
		% schliese alle Gruppen und berechne approximierte, logarithmisch abgeschwächte Abstoßungsfaktor anhand von benötigter Größe
		% berechne Layout auf oberster Stufe
		% Lege Ports für Gruppen auf oberster Stufe fest
		
		% berechne iterativ absteigend Layouts in Gruppen mit festgelegten Ports

% Idee für Verhalten: Anker
% was passiert, wenn Gruppe geöffnet oder geschlossen wird?
% - jedes Element auf oberster Ebene bekommt festen aber schwachen Anker an Startposition von Anfangslayout um immer dazu ähnlich zu bleiben
% - jedes Element auf gleicher Ebene wie veränderte Gruppe bekommt einen Anker an aktueller Position
	% geöffnete Gruppe stoßt alle Elemente ab, sodass sie genug Platz hat.  Rest auch kräftebasiert um Ähnlichkeit zu Layout davor zu bewahren
	% geschlossene Gruppe stoßt nun Elemente weniger ab. Rest wieder kräftebasiert um Ähnlichkeit zu Layout davor zu bewahren
% - Ist geänderte Gruppe nicht auf oberster Ebene, so wiederholt sich der Prozess iterativ nach oben, da übergeordnete Gruppen auch mehr Platz brauen oder freigeben

% - Erweiterungsmöglichkeit: Anker von jedem Schritt speichern, jedoch pro Schritt abschwächen

\section{Evaluation und Vergleich zu anderen Lösungsansätzen}
% zu vergleichen mit hierarchischem Layout und alles kräftebasiert

% gesammelte Argumente
	% KB alles kräftebasiert, HS hierarchisch/''Stufen''-Layout
% + Gruppen vorallem semantisch organisiert, weniger strukturell als bei KB
% + Änderungskonstanz der Gruppen (innerhalb von Gruppe) als bei KB
% - komplexer als KB
% + Gruppen überschneiden sich nicht wie bzw einfach zu getrennt zu halten als bei KB 
% + HS ist nur ohne Zykel gut umsetzbar
% + komptakter als HS
% - Struktur evtl nicht so sichtbar wie bei HS aber auch wie bei KB
% + iterativ anwendbar
% + schöner :)
% o Gruppen im Vordergrund der Darstellung durch Ports, nicht Knoten
% o Kantenrounting nicht unbedingt trivial
% 

\section{Quellen}
\bibliography{SeminarAlgoQuellen}

\section*{About}
Entstanden im Rahmen des Seminars Visualisierung komplexer Argumentation bzw. Algorithmen zur Visualisierung von Debatten 
am Karlsruher Institut für Technology im Wintersemester 2014-15 unter der Leitung von 
Jun.-Prof. Gregor Betz, Diplom Inform. Andreas Gemsa und Dr. Ignaz Rutter.


%--END--END--END--END--END--END--END--END--END--END--END--
\end{document}
%--END--END--END--END--END--END--END--END--END--END--END--